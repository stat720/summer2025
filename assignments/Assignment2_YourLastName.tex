% Options for packages loaded elsewhere
\PassOptionsToPackage{unicode}{hyperref}
\PassOptionsToPackage{hyphens}{url}
%
\documentclass[
]{article}
\usepackage{amsmath,amssymb}
\usepackage{iftex}
\ifPDFTeX
  \usepackage[T1]{fontenc}
  \usepackage[utf8]{inputenc}
  \usepackage{textcomp} % provide euro and other symbols
\else % if luatex or xetex
  \usepackage{unicode-math} % this also loads fontspec
  \defaultfontfeatures{Scale=MatchLowercase}
  \defaultfontfeatures[\rmfamily]{Ligatures=TeX,Scale=1}
\fi
\usepackage{lmodern}
\ifPDFTeX\else
  % xetex/luatex font selection
\fi
% Use upquote if available, for straight quotes in verbatim environments
\IfFileExists{upquote.sty}{\usepackage{upquote}}{}
\IfFileExists{microtype.sty}{% use microtype if available
  \usepackage[]{microtype}
  \UseMicrotypeSet[protrusion]{basicmath} % disable protrusion for tt fonts
}{}
\makeatletter
\@ifundefined{KOMAClassName}{% if non-KOMA class
  \IfFileExists{parskip.sty}{%
    \usepackage{parskip}
  }{% else
    \setlength{\parindent}{0pt}
    \setlength{\parskip}{6pt plus 2pt minus 1pt}}
}{% if KOMA class
  \KOMAoptions{parskip=half}}
\makeatother
\usepackage{xcolor}
\usepackage[margin=1in]{geometry}
\usepackage{longtable,booktabs,array}
\usepackage{calc} % for calculating minipage widths
% Correct order of tables after \paragraph or \subparagraph
\usepackage{etoolbox}
\makeatletter
\patchcmd\longtable{\par}{\if@noskipsec\mbox{}\fi\par}{}{}
\makeatother
% Allow footnotes in longtable head/foot
\IfFileExists{footnotehyper.sty}{\usepackage{footnotehyper}}{\usepackage{footnote}}
\makesavenoteenv{longtable}
\usepackage{graphicx}
\makeatletter
\newsavebox\pandoc@box
\newcommand*\pandocbounded[1]{% scales image to fit in text height/width
  \sbox\pandoc@box{#1}%
  \Gscale@div\@tempa{\textheight}{\dimexpr\ht\pandoc@box+\dp\pandoc@box\relax}%
  \Gscale@div\@tempb{\linewidth}{\wd\pandoc@box}%
  \ifdim\@tempb\p@<\@tempa\p@\let\@tempa\@tempb\fi% select the smaller of both
  \ifdim\@tempa\p@<\p@\scalebox{\@tempa}{\usebox\pandoc@box}%
  \else\usebox{\pandoc@box}%
  \fi%
}
% Set default figure placement to htbp
\def\fps@figure{htbp}
\makeatother
\setlength{\emergencystretch}{3em} % prevent overfull lines
\providecommand{\tightlist}{%
  \setlength{\itemsep}{0pt}\setlength{\parskip}{0pt}}
\setcounter{secnumdepth}{-\maxdimen} % remove section numbering
\usepackage{bookmark}
\IfFileExists{xurl.sty}{\usepackage{xurl}}{} % add URL line breaks if available
\urlstyle{same}
\hypersetup{
  pdftitle={Assignment 2},
  pdfauthor={Your Name},
  hidelinks,
  pdfcreator={LaTeX via pandoc}}

\title{Assignment 2}
\author{Your Name}
\date{2025-06-20}

\begin{document}
\maketitle

\subsection{About this assignment}\label{about-this-assignment}

\begin{itemize}
\tightlist
\item
  \textbf{Goal:} to review the concepts learned about treatment
  structure and design structure of an experiment.
\item
  \textbf{Due:} Wednesday, July 2.
\end{itemize}

\subsection{1. A group of agricultural scientists have received funding
to study the effects of water availability and nitrogen fertilizer on
corn yield. They know that the effect of nitrogen may depend on the
water availability and they have enough resources to study 3 irrigation
regimes that mimick dry years, average years, and wet years. At the same
time, they wish to study three levels of Nitrogen fertilizer: the
farmer's most common rate, farmer's + 200 lb N/ac, and farmer's - 200 lb
N/ac. The scientists can run this experiment in a field that
historically has always showed very low spatial
variability.}\label{a-group-of-agricultural-scientists-have-received-funding-to-study-the-effects-of-water-availability-and-nitrogen-fertilizer-on-corn-yield.-they-know-that-the-effect-of-nitrogen-may-depend-on-the-water-availability-and-they-have-enough-resources-to-study-3-irrigation-regimes-that-mimick-dry-years-average-years-and-wet-years.-at-the-same-time-they-wish-to-study-three-levels-of-nitrogen-fertilizer-the-farmers-most-common-rate-farmers-200-lb-nac-and-farmers---200-lb-nac.-the-scientists-can-run-this-experiment-in-a-field-that-historically-has-always-showed-very-low-spatial-variability.}

\subsubsection{a. What treatment structure would you use? How many
levels for each
factor?}\label{a.-what-treatment-structure-would-you-use-how-many-levels-for-each-factor}

\subsubsection{b. What design structure would you
use?}\label{b.-what-design-structure-would-you-use}

\subsubsection{c.~Fill out the following table with the values you
consider:}\label{c.-fill-out-the-following-table-with-the-values-you-consider}

\begin{longtable}[]{@{}ll@{}}
\toprule\noalign{}
Source & df \\
\midrule\noalign{}
\endhead
\bottomrule\noalign{}
\endlastfoot
Factor A & fill out \\
Factor B & fill out \\
A x B & fill out \\
Error & fill out \\
Total & fill out \\
\end{longtable}

\subsubsection{d.~Mention two strategies to increase the degrees of
freedom of the error (see table above), show how they would increase
said degrees of freedom, and discuss the implications in the conclusions
of the
study.}\label{d.-mention-two-strategies-to-increase-the-degrees-of-freedom-of-the-error-see-table-above-show-how-they-would-increase-said-degrees-of-freedom-and-discuss-the-implications-in-the-conclusions-of-the-study.}

\subsection{2. Find a publication from your domain (e.g., animal
science, agronomy, etc.) and paste a snippet of the Materials and
Methods section where they describe the designed experiment. Answer the
following
questions:}\label{find-a-publication-from-your-domain-e.g.-animal-science-agronomy-etc.-and-paste-a-snippet-of-the-materials-and-methods-section-where-they-describe-the-designed-experiment.-answer-the-following-questions}

\subsubsection{a. What is the treatment structure? How many levels for
each
factor?}\label{a.-what-is-the-treatment-structure-how-many-levels-for-each-factor}

\subsubsection{b. What is the design
structure?}\label{b.-what-is-the-design-structure}

\subsubsection{c.~Would you change anything in the wording/writing to
make it more
understandable?}\label{c.-would-you-change-anything-in-the-wordingwriting-to-make-it-more-understandable}

\end{document}
