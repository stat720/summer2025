% Options for packages loaded elsewhere
\PassOptionsToPackage{unicode}{hyperref}
\PassOptionsToPackage{hyphens}{url}
%
\documentclass[
]{article}
\usepackage{amsmath,amssymb}
\usepackage{iftex}
\ifPDFTeX
  \usepackage[T1]{fontenc}
  \usepackage[utf8]{inputenc}
  \usepackage{textcomp} % provide euro and other symbols
\else % if luatex or xetex
  \usepackage{unicode-math} % this also loads fontspec
  \defaultfontfeatures{Scale=MatchLowercase}
  \defaultfontfeatures[\rmfamily]{Ligatures=TeX,Scale=1}
\fi
\usepackage{lmodern}
\ifPDFTeX\else
  % xetex/luatex font selection
\fi
% Use upquote if available, for straight quotes in verbatim environments
\IfFileExists{upquote.sty}{\usepackage{upquote}}{}
\IfFileExists{microtype.sty}{% use microtype if available
  \usepackage[]{microtype}
  \UseMicrotypeSet[protrusion]{basicmath} % disable protrusion for tt fonts
}{}
\makeatletter
\@ifundefined{KOMAClassName}{% if non-KOMA class
  \IfFileExists{parskip.sty}{%
    \usepackage{parskip}
  }{% else
    \setlength{\parindent}{0pt}
    \setlength{\parskip}{6pt plus 2pt minus 1pt}}
}{% if KOMA class
  \KOMAoptions{parskip=half}}
\makeatother
\usepackage{xcolor}
\usepackage[margin=1in]{geometry}
\usepackage{color}
\usepackage{fancyvrb}
\newcommand{\VerbBar}{|}
\newcommand{\VERB}{\Verb[commandchars=\\\{\}]}
\DefineVerbatimEnvironment{Highlighting}{Verbatim}{commandchars=\\\{\}}
% Add ',fontsize=\small' for more characters per line
\usepackage{framed}
\definecolor{shadecolor}{RGB}{248,248,248}
\newenvironment{Shaded}{\begin{snugshade}}{\end{snugshade}}
\newcommand{\AlertTok}[1]{\textcolor[rgb]{0.94,0.16,0.16}{#1}}
\newcommand{\AnnotationTok}[1]{\textcolor[rgb]{0.56,0.35,0.01}{\textbf{\textit{#1}}}}
\newcommand{\AttributeTok}[1]{\textcolor[rgb]{0.13,0.29,0.53}{#1}}
\newcommand{\BaseNTok}[1]{\textcolor[rgb]{0.00,0.00,0.81}{#1}}
\newcommand{\BuiltInTok}[1]{#1}
\newcommand{\CharTok}[1]{\textcolor[rgb]{0.31,0.60,0.02}{#1}}
\newcommand{\CommentTok}[1]{\textcolor[rgb]{0.56,0.35,0.01}{\textit{#1}}}
\newcommand{\CommentVarTok}[1]{\textcolor[rgb]{0.56,0.35,0.01}{\textbf{\textit{#1}}}}
\newcommand{\ConstantTok}[1]{\textcolor[rgb]{0.56,0.35,0.01}{#1}}
\newcommand{\ControlFlowTok}[1]{\textcolor[rgb]{0.13,0.29,0.53}{\textbf{#1}}}
\newcommand{\DataTypeTok}[1]{\textcolor[rgb]{0.13,0.29,0.53}{#1}}
\newcommand{\DecValTok}[1]{\textcolor[rgb]{0.00,0.00,0.81}{#1}}
\newcommand{\DocumentationTok}[1]{\textcolor[rgb]{0.56,0.35,0.01}{\textbf{\textit{#1}}}}
\newcommand{\ErrorTok}[1]{\textcolor[rgb]{0.64,0.00,0.00}{\textbf{#1}}}
\newcommand{\ExtensionTok}[1]{#1}
\newcommand{\FloatTok}[1]{\textcolor[rgb]{0.00,0.00,0.81}{#1}}
\newcommand{\FunctionTok}[1]{\textcolor[rgb]{0.13,0.29,0.53}{\textbf{#1}}}
\newcommand{\ImportTok}[1]{#1}
\newcommand{\InformationTok}[1]{\textcolor[rgb]{0.56,0.35,0.01}{\textbf{\textit{#1}}}}
\newcommand{\KeywordTok}[1]{\textcolor[rgb]{0.13,0.29,0.53}{\textbf{#1}}}
\newcommand{\NormalTok}[1]{#1}
\newcommand{\OperatorTok}[1]{\textcolor[rgb]{0.81,0.36,0.00}{\textbf{#1}}}
\newcommand{\OtherTok}[1]{\textcolor[rgb]{0.56,0.35,0.01}{#1}}
\newcommand{\PreprocessorTok}[1]{\textcolor[rgb]{0.56,0.35,0.01}{\textit{#1}}}
\newcommand{\RegionMarkerTok}[1]{#1}
\newcommand{\SpecialCharTok}[1]{\textcolor[rgb]{0.81,0.36,0.00}{\textbf{#1}}}
\newcommand{\SpecialStringTok}[1]{\textcolor[rgb]{0.31,0.60,0.02}{#1}}
\newcommand{\StringTok}[1]{\textcolor[rgb]{0.31,0.60,0.02}{#1}}
\newcommand{\VariableTok}[1]{\textcolor[rgb]{0.00,0.00,0.00}{#1}}
\newcommand{\VerbatimStringTok}[1]{\textcolor[rgb]{0.31,0.60,0.02}{#1}}
\newcommand{\WarningTok}[1]{\textcolor[rgb]{0.56,0.35,0.01}{\textbf{\textit{#1}}}}
\usepackage{graphicx}
\makeatletter
\newsavebox\pandoc@box
\newcommand*\pandocbounded[1]{% scales image to fit in text height/width
  \sbox\pandoc@box{#1}%
  \Gscale@div\@tempa{\textheight}{\dimexpr\ht\pandoc@box+\dp\pandoc@box\relax}%
  \Gscale@div\@tempb{\linewidth}{\wd\pandoc@box}%
  \ifdim\@tempb\p@<\@tempa\p@\let\@tempa\@tempb\fi% select the smaller of both
  \ifdim\@tempa\p@<\p@\scalebox{\@tempa}{\usebox\pandoc@box}%
  \else\usebox{\pandoc@box}%
  \fi%
}
% Set default figure placement to htbp
\def\fps@figure{htbp}
\makeatother
\setlength{\emergencystretch}{3em} % prevent overfull lines
\providecommand{\tightlist}{%
  \setlength{\itemsep}{0pt}\setlength{\parskip}{0pt}}
\setcounter{secnumdepth}{-\maxdimen} % remove section numbering
\usepackage{bookmark}
\IfFileExists{xurl.sty}{\usepackage{xurl}}{} % add URL line breaks if available
\urlstyle{same}
\hypersetup{
  pdftitle={Assignment 1},
  pdfauthor={Test Student},
  hidelinks,
  pdfcreator={LaTeX via pandoc}}

\title{Assignment 1}
\author{Test Student}
\date{2025-06-13}

\begin{document}
\maketitle

\subsection{Goal of this assignment}\label{goal-of-this-assignment}

The goal of this assignment is to (i) review a few basic concepts and to
(ii) make sure that you can knit an \texttt{Rmd} document with the basic
features that will be needed throughout this course. Please complete
exercises (1) and (2), rename the \texttt{Rmd} file to
``Assignment1\_YourLastName.Rmd'' (e.g., ``Assignment1\_Smith.Rmd''),
your name in the header, and knit the Rmd to an html file or pdf file.
Please submit that html or pdf file on CANVAS by Friday, June 20th
midnight. You may work in pairs, but each one of you will have to submit
your own file.

\subsection{1. Using your own words, describe a randomized complete
block design (RCBD) in 3-5 sentences. Then write out a statistical model
(using mathematical notation) that could be fitted to data generated by
an
RCBD.}\label{using-your-own-words-describe-a-randomized-complete-block-design-rcbd-in-3-5-sentences.-then-write-out-a-statistical-model-using-mathematical-notation-that-could-be-fitted-to-data-generated-by-an-rcbd.}

An RCBD is a type of design that gathers groups of similar EUs and
accounts for their variability\ldots.

A statistical model that could be connected to an RCBD is

\[y_{ijk} = \mu + T_j + B_k + \varepsilon_{ijk}, \varepsilon_{ijk}\sim N(0, \sigma^2),\ \]

where: - \(\mu\) is the overall mean, - \(T_j\) is the effect of the
\(j\)th treatment, - \(B_k\) is the effect of the \(k\)th block, -
\(\varepsilon_{ijk}\) is the residual of the \(i\)th observation,
\(j\)th treatment, \(k\)th block.

\subsection{2. Edit the R code below:}\label{edit-the-r-code-below}

\begin{itemize}
\tightlist
\item
  Silence the \texttt{messages} and \texttt{warnings}: your submitted
  pdf or html should include the code, but not the warnings/messages
  seen below.
\item
  Print a summary of \texttt{model1}.
\item
  Find and print the value for \(\hat\sigma^2\) estimated in
  \texttt{model1}.
\end{itemize}

\begin{Shaded}
\begin{Highlighting}[]
\FunctionTok{library}\NormalTok{(tidyverse)}
\FunctionTok{library}\NormalTok{(agridat)}
\FunctionTok{data}\NormalTok{(}\StringTok{"omer.sorghum"}\NormalTok{)}
\NormalTok{df }\OtherTok{\textless{}{-}}\NormalTok{ omer.sorghum }\SpecialCharTok{\%\textgreater{}\%} \FunctionTok{filter}\NormalTok{(env }\SpecialCharTok{==} \StringTok{"E3"}\NormalTok{)}
\NormalTok{model1 }\OtherTok{\textless{}{-}} \FunctionTok{lm}\NormalTok{(yield }\SpecialCharTok{\textasciitilde{}} \DecValTok{1} \SpecialCharTok{+}\NormalTok{ gen }\SpecialCharTok{+}\NormalTok{ rep, }\AttributeTok{data =}\NormalTok{ df)}
\end{Highlighting}
\end{Shaded}

\begin{Shaded}
\begin{Highlighting}[]
\FunctionTok{summary}\NormalTok{(model1)}
\end{Highlighting}
\end{Shaded}

\begin{verbatim}
## 
## Call:
## lm(formula = yield ~ 1 + gen + rep, data = df)
## 
## Residuals:
##     Min      1Q  Median      3Q     Max 
## -349.66  -81.58   -1.08   78.47  318.59 
## 
## Coefficients:
##             Estimate Std. Error t value Pr(>|t|)    
## (Intercept)   672.03      86.46   7.773 3.29e-10 ***
## genG02        -66.52     113.20  -0.588 0.559393    
## genG03        223.07     113.20   1.971 0.054201 .  
## genG04        167.40     113.20   1.479 0.145336    
## genG05         61.05     113.20   0.539 0.591996    
## genG06       -267.20     113.20  -2.361 0.022111 *  
## genG07        355.20     113.20   3.138 0.002826 ** 
## genG08        159.53     113.20   1.409 0.164820    
## genG09        231.31     113.20   2.043 0.046198 *  
## genG10        156.66     113.20   1.384 0.172393    
## genG11        146.29     113.20   1.292 0.202071    
## genG12        -42.87     113.20  -0.379 0.706453    
## genG13        243.18     113.20   2.148 0.036467 *  
## genG14          1.06     113.20   0.009 0.992565    
## genG15         64.72     113.20   0.572 0.570007    
## genG16         22.81     113.20   0.201 0.841122    
## genG17       -296.66     113.20  -2.621 0.011534 *  
## genG18        448.28     113.20   3.960 0.000233 ***
## repR2        -150.83      53.36  -2.827 0.006704 ** 
## repR3         -85.66      53.36  -1.605 0.114592    
## repR4        -124.21      53.36  -2.328 0.023936 *  
## ---
## Signif. codes:  0 '***' 0.001 '**' 0.01 '*' 0.05 '.' 0.1 ' ' 1
## 
## Residual standard error: 160.1 on 51 degrees of freedom
## Multiple R-squared:  0.6748, Adjusted R-squared:  0.5473 
## F-statistic: 5.291 on 20 and 51 DF,  p-value: 7.64e-07
\end{verbatim}

\begin{Shaded}
\begin{Highlighting}[]
\FunctionTok{sigma}\NormalTok{(model1)}\SpecialCharTok{\^{}}\DecValTok{2}
\end{Highlighting}
\end{Shaded}

\begin{verbatim}
## [1] 25627.36
\end{verbatim}

\begin{Shaded}
\begin{Highlighting}[]
\FunctionTok{sum}\NormalTok{(model1}\SpecialCharTok{$}\NormalTok{residuals}\SpecialCharTok{\^{}}\DecValTok{2}\NormalTok{) }\SpecialCharTok{/}\NormalTok{ model1}\SpecialCharTok{$}\NormalTok{df.residual}
\end{Highlighting}
\end{Shaded}

\begin{verbatim}
## [1] 25627.36
\end{verbatim}

\begin{Shaded}
\begin{Highlighting}[]
\FunctionTok{summary}\NormalTok{(model1)}\SpecialCharTok{$}\NormalTok{sigma}\SpecialCharTok{\^{}}\DecValTok{2}
\end{Highlighting}
\end{Shaded}

\begin{verbatim}
## [1] 25627.36
\end{verbatim}

\end{document}
